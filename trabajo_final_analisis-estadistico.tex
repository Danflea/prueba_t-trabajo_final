% Options for packages loaded elsewhere
\PassOptionsToPackage{unicode}{hyperref}
\PassOptionsToPackage{hyphens}{url}
%
\documentclass[
]{article}
\usepackage{amsmath,amssymb}
\usepackage{iftex}
\ifPDFTeX
  \usepackage[T1]{fontenc}
  \usepackage[utf8]{inputenc}
  \usepackage{textcomp} % provide euro and other symbols
\else % if luatex or xetex
  \usepackage{unicode-math} % this also loads fontspec
  \defaultfontfeatures{Scale=MatchLowercase}
  \defaultfontfeatures[\rmfamily]{Ligatures=TeX,Scale=1}
\fi
\usepackage{lmodern}
\ifPDFTeX\else
  % xetex/luatex font selection
\fi
% Use upquote if available, for straight quotes in verbatim environments
\IfFileExists{upquote.sty}{\usepackage{upquote}}{}
\IfFileExists{microtype.sty}{% use microtype if available
  \usepackage[]{microtype}
  \UseMicrotypeSet[protrusion]{basicmath} % disable protrusion for tt fonts
}{}
\makeatletter
\@ifundefined{KOMAClassName}{% if non-KOMA class
  \IfFileExists{parskip.sty}{%
    \usepackage{parskip}
  }{% else
    \setlength{\parindent}{0pt}
    \setlength{\parskip}{6pt plus 2pt minus 1pt}}
}{% if KOMA class
  \KOMAoptions{parskip=half}}
\makeatother
\usepackage{xcolor}
\usepackage[margin=1in]{geometry}
\usepackage{color}
\usepackage{fancyvrb}
\newcommand{\VerbBar}{|}
\newcommand{\VERB}{\Verb[commandchars=\\\{\}]}
\DefineVerbatimEnvironment{Highlighting}{Verbatim}{commandchars=\\\{\}}
% Add ',fontsize=\small' for more characters per line
\usepackage{framed}
\definecolor{shadecolor}{RGB}{248,248,248}
\newenvironment{Shaded}{\begin{snugshade}}{\end{snugshade}}
\newcommand{\AlertTok}[1]{\textcolor[rgb]{0.94,0.16,0.16}{#1}}
\newcommand{\AnnotationTok}[1]{\textcolor[rgb]{0.56,0.35,0.01}{\textbf{\textit{#1}}}}
\newcommand{\AttributeTok}[1]{\textcolor[rgb]{0.13,0.29,0.53}{#1}}
\newcommand{\BaseNTok}[1]{\textcolor[rgb]{0.00,0.00,0.81}{#1}}
\newcommand{\BuiltInTok}[1]{#1}
\newcommand{\CharTok}[1]{\textcolor[rgb]{0.31,0.60,0.02}{#1}}
\newcommand{\CommentTok}[1]{\textcolor[rgb]{0.56,0.35,0.01}{\textit{#1}}}
\newcommand{\CommentVarTok}[1]{\textcolor[rgb]{0.56,0.35,0.01}{\textbf{\textit{#1}}}}
\newcommand{\ConstantTok}[1]{\textcolor[rgb]{0.56,0.35,0.01}{#1}}
\newcommand{\ControlFlowTok}[1]{\textcolor[rgb]{0.13,0.29,0.53}{\textbf{#1}}}
\newcommand{\DataTypeTok}[1]{\textcolor[rgb]{0.13,0.29,0.53}{#1}}
\newcommand{\DecValTok}[1]{\textcolor[rgb]{0.00,0.00,0.81}{#1}}
\newcommand{\DocumentationTok}[1]{\textcolor[rgb]{0.56,0.35,0.01}{\textbf{\textit{#1}}}}
\newcommand{\ErrorTok}[1]{\textcolor[rgb]{0.64,0.00,0.00}{\textbf{#1}}}
\newcommand{\ExtensionTok}[1]{#1}
\newcommand{\FloatTok}[1]{\textcolor[rgb]{0.00,0.00,0.81}{#1}}
\newcommand{\FunctionTok}[1]{\textcolor[rgb]{0.13,0.29,0.53}{\textbf{#1}}}
\newcommand{\ImportTok}[1]{#1}
\newcommand{\InformationTok}[1]{\textcolor[rgb]{0.56,0.35,0.01}{\textbf{\textit{#1}}}}
\newcommand{\KeywordTok}[1]{\textcolor[rgb]{0.13,0.29,0.53}{\textbf{#1}}}
\newcommand{\NormalTok}[1]{#1}
\newcommand{\OperatorTok}[1]{\textcolor[rgb]{0.81,0.36,0.00}{\textbf{#1}}}
\newcommand{\OtherTok}[1]{\textcolor[rgb]{0.56,0.35,0.01}{#1}}
\newcommand{\PreprocessorTok}[1]{\textcolor[rgb]{0.56,0.35,0.01}{\textit{#1}}}
\newcommand{\RegionMarkerTok}[1]{#1}
\newcommand{\SpecialCharTok}[1]{\textcolor[rgb]{0.81,0.36,0.00}{\textbf{#1}}}
\newcommand{\SpecialStringTok}[1]{\textcolor[rgb]{0.31,0.60,0.02}{#1}}
\newcommand{\StringTok}[1]{\textcolor[rgb]{0.31,0.60,0.02}{#1}}
\newcommand{\VariableTok}[1]{\textcolor[rgb]{0.00,0.00,0.00}{#1}}
\newcommand{\VerbatimStringTok}[1]{\textcolor[rgb]{0.31,0.60,0.02}{#1}}
\newcommand{\WarningTok}[1]{\textcolor[rgb]{0.56,0.35,0.01}{\textbf{\textit{#1}}}}
\usepackage{graphicx}
\makeatletter
\def\maxwidth{\ifdim\Gin@nat@width>\linewidth\linewidth\else\Gin@nat@width\fi}
\def\maxheight{\ifdim\Gin@nat@height>\textheight\textheight\else\Gin@nat@height\fi}
\makeatother
% Scale images if necessary, so that they will not overflow the page
% margins by default, and it is still possible to overwrite the defaults
% using explicit options in \includegraphics[width, height, ...]{}
\setkeys{Gin}{width=\maxwidth,height=\maxheight,keepaspectratio}
% Set default figure placement to htbp
\makeatletter
\def\fps@figure{htbp}
\makeatother
\setlength{\emergencystretch}{3em} % prevent overfull lines
\providecommand{\tightlist}{%
  \setlength{\itemsep}{0pt}\setlength{\parskip}{0pt}}
\setcounter{secnumdepth}{-\maxdimen} % remove section numbering
% definitions for citeproc citations
\NewDocumentCommand\citeproctext{}{}
\NewDocumentCommand\citeproc{mm}{%
  \begingroup\def\citeproctext{#2}\cite{#1}\endgroup}
\makeatletter
 % allow citations to break across lines
 \let\@cite@ofmt\@firstofone
 % avoid brackets around text for \cite:
 \def\@biblabel#1{}
 \def\@cite#1#2{{#1\if@tempswa , #2\fi}}
\makeatother
\newlength{\cslhangindent}
\setlength{\cslhangindent}{1.5em}
\newlength{\csllabelwidth}
\setlength{\csllabelwidth}{3em}
\newenvironment{CSLReferences}[2] % #1 hanging-indent, #2 entry-spacing
 {\begin{list}{}{%
  \setlength{\itemindent}{0pt}
  \setlength{\leftmargin}{0pt}
  \setlength{\parsep}{0pt}
  % turn on hanging indent if param 1 is 1
  \ifodd #1
   \setlength{\leftmargin}{\cslhangindent}
   \setlength{\itemindent}{-1\cslhangindent}
  \fi
  % set entry spacing
  \setlength{\itemsep}{#2\baselineskip}}}
 {\end{list}}
\usepackage{calc}
\newcommand{\CSLBlock}[1]{\hfill\break\parbox[t]{\linewidth}{\strut\ignorespaces#1\strut}}
\newcommand{\CSLLeftMargin}[1]{\parbox[t]{\csllabelwidth}{\strut#1\strut}}
\newcommand{\CSLRightInline}[1]{\parbox[t]{\linewidth - \csllabelwidth}{\strut#1\strut}}
\newcommand{\CSLIndent}[1]{\hspace{\cslhangindent}#1}
\usepackage{calligra}
\usepackage{ragged2e}
\justifying
\usepackage{tocbibind}
\usepackage{booktabs}
\usepackage{longtable}
\usepackage{array}
\usepackage{multirow}
\usepackage{wrapfig}
\usepackage{float}
\usepackage{colortbl}
\usepackage{pdflscape}
\usepackage{tabu}
\usepackage{threeparttable}
\usepackage{threeparttablex}
\usepackage[normalem]{ulem}
\usepackage{makecell}
\usepackage{xcolor}
\ifLuaTeX
  \usepackage{selnolig}  % disable illegal ligatures
\fi
\usepackage{bookmark}
\IfFileExists{xurl.sty}{\usepackage{xurl}}{} % add URL line breaks if available
\urlstyle{same}
\hypersetup{
  hidelinks,
  pdfcreator={LaTeX via pandoc}}

\author{}
\date{\vspace{-2.5em}}

\begin{document}

\begin{titlepage}
\begin{center}
\vspace*{\baselineskip}

{\bf\fontsize{19}{0}{\selectfont{ UNIVERSIDAD NACIONAL   AGRARIA \\ [0.5cm] LA MOLINA}}\\[0.5cm]
\fontsize{10}{0}{DEPARTAMENTO ACADÉMICO ESTADÍSTICA E INFORMÁTICA}\\ [0.5cm]
\bf\fontsize{14}{0}{\selectfont{ CARRERA PROFESIONAL DE ESTADÍSTICA E INFORMÁTICA}}
}\\[0.5cm]
\end{center}
\begin{center}
\vspace*{0.3in}
\begin{figure}[htb]
\begin{center}
\includegraphics[width=4.2cm,height=5cm]{imagen.jpg}
\end{center}
\end{figure}
\vspace*{0.1in}
\begin{Large}
\textbf{Prueba t de student para \\  muestras independientes } \\
\end{Large}
\vspace{0.4cm}
\begin{large}

\begin{center}
\textbf{Integrantes:}

\begin{tabbing}
    \hspace{2cm} Estrella Guerra, Danilo David \quad \= \quad         20220763 \\
    \hspace{2cm} Gómez Vigo, Héctor Estéfano   \quad \= \quad      20230397 \\
    \hspace{2cm} Montufar Paiva, Yeraldi Mercedes \quad  20230400 \\
    \hspace{2cm} Rojas Taco, Fabiana Romina    \quad \= \quad      20220956 \\
    \hspace{2cm} Zavala Malpartida, Kay Daniela L. \quad 20230420 \\
\end{tabbing}

\end{center}

\end{large}

\vspace*{0.2in}
\begin{large}
\textbf{Profesor:} \\
\end{large}
\vspace*{0.01in}

\begin{large}
Mg. Samuel HUAMANÍ  FLORES\\ [0.4cm]
\end{large}


\vspace*{0.3in}
\begin{large}
\textbf{LIMA-PERÚ }\\ [0.5cm]
\textbf{2024}
\end{large}
\end{center}
\end{titlepage}

\newpage

\renewcommand{\contentsname}{Indice de Contenido}

\tableofcontents

\newpage

\section*{Introducción}\label{introducciuxf3n}
\addcontentsline{toc}{section}{Introducción}

La educación es uno de los pilares que sostienen la base de una sociedad
justa y equitativa. Esto significa que el acceso a una educación
secundaria adecuada es uno de los factores que realmente importan en el
desarrollo personal y profesional de un adolescente. Este es uno de los
mayores desafíos y las principales inequidades persistentes en el Perú,
considerando el acceso y la calidad de la educación. El Estado aquí ha
tratado de poner en acción varias políticas para mejorar la cobertura y
la calidad de la educación; sin embargo, la brecha se muestra claramente
entre las áreas rurales y urbanas. Según UNICEF, un poco más de un
tercio de todos los niños de las áreas rurales de los países en
desarrollo logran alcanzar los niveles mínimos de competencia en las
competencias básicas de lectura y matemáticas, lo que refleja
profundamente las profundas desigualdades del sistema educativo. Esto es
particularmente relevante para una región como Ayacucho, donde las
barreras geográficas, sociales y económicas tienen impactos adversos en
el rendimiento escolar.

La educación como medio de progreso en Ayacucho ``El incremento se
evidencia con una sólida historia de conflictos y desigualdades en
Ayacucho, más significativa en los últimos años''. Entre 2007 y 2015, se
evidenció un aumento en términos de matrícula y retención de estudiantes
en el nivel de educación secundaria. Sin embargo, persisten grandes
disparidades en los rendimientos académicos de los dos lugares: áreas
rurales y urbanas. Factores como que los estudiantes rurales han seguido
enfrentando dificultades y desafíos en los aspectos más simples, como el
acceso limitado a materiales educativos, la infraestructura deficiente
en las escuelas y la falta de capacitación del personal docente. Estos
han limitado el aprendizaje de calidad y han dificultado que los jóvenes
desarrollen todo su potencial.

Cabe señalar que la educación no sólo empodera a un individuo, sino que
también tiene la implicación más profunda para el crecimiento económico
y social de una comunidad. En esta área principalmente rural, las
barreras educativas en Ayacucho solo han tenido el propósito de
perpetuar el ciclo de pobreza y, por lo tanto, limitar el potencial de
progreso de sus estudiantes.

Esta relación entre el nivel educativo y el desarrollo regional está
bien documentada por muchos autores que coinciden en que la inversión en
educación es crucial para luchar contra la pobreza y fomentar el
crecimiento sostenible. UNICEF (2016, p.~40). Las desigualdades actuales
en la calidad de la educación entre los entornos urbanos y rurales
niegan a los niños y jóvenes las mismas oportunidades de recibir una
buena educación, lo que exige una corrección inmediata de la situación.

Con estos antecedentes, el presente estudio se centrará principalmente
en el análisis del nivel de educación en el segundo ciclo de la
educación secundaria en Ayacucho, distinguiendo entre las zonas urbanas
y rurales. El objetivo general del presente estudio es identificar
disparidades generales llamativas en el logro de aprendizaje y los
factores que las explican. Como tal, esta investigación podría servir
para proporcionar una visión mucho más clara de la situación de la
educación en Ayacucho y poder sugerir recomendaciones que apoyen la
mejora de las condiciones educativas para las áreas más vulnerables
dentro de la provincia.

\newpage

\section*{Planteamiento del Problema}\label{planteamiento-del-problema}
\addcontentsline{toc}{section}{Planteamiento del Problema}

La calidad del aprendizaje sigue siendo una preocupación en la mayor
parte del país, como Ayacucho, incluso cuando se dan pasos hacia el
aumento de la cobertura en educación en Perú. De hecho, con respecto al
informe Ayacucho Cómo Vamos en Educación 2022, es evidente el hecho de
que existe un marcado contraste en los logros académicos entre los
estudiantes que viven en zonas urbanas y los que residen en las zonas
rurales. Esto no solo señala problemas estructurales en términos de
infraestructura y acceso a recursos educativos sino también el creciente
grado de vulnerabilidad social y económica que atraviesan los
estudiantes rurales. Por ejemplo, el informe sostiene que ``las
deficiencias en la calidad educativa también se manifiestan en los
resultados de las pruebas estandarizadas en las que los estudiantes
rurales obtienen puntuaciones marcadamente más bajas que sus
contrapartes urbanas'' (p.~24).

Existe una enorme brecha en la comprensión lectora y las habilidades
matemáticas de los estudiantes entre los estudiantes rurales y urbanos
en Ayacucho. Las zonas rurales son aquellas de muy bajo rendimiento,
ocasionado por los siguientes factores: falta de acceso a
infraestructura tecnológica, escasez de docentes calificados y el escaso
involucramiento de los padres en la educación de sus hijos (Ayacucho
Cómo Vamos en Educación, 2022, p.~26).

Estos factores profundizan la brecha en educación, haciendo más dura y
directa la desigualdad, y a su vez, perjudicando las oportunidades de
desarrollo de los educandos rurales. UNICEF ha señalado que millones de
niños en todo el mundo no reciben igualdad de oportunidades para una
educación de calidad, de lo cual Ayacucho no es la excepción. Como lo
expresa UNICEF, esto es ``una violación de la Convención sobre los
Derechos del Niño'' (Perú (2016), 2016, p.~28). Explica El estudio del
nivel de educación en Ayacucho es importante no solo para entender las
diferencias en el rendimiento académico sino también porque se abordan
con medios de política pública. Se espera que el análisis de las
disparidades educativas tenga un impacto en la formulación de
intervenciones sensibles a las necesidades específicas de los
estudiantes rurales. El otro punto que se abordará es la propuesta de
mejora de la calidad en relación a estas desigualdades, y se asegurará
que todos los estudiantes, sin importar el entorno en el que se
encuentren, tengan oportunidades similares de aprendizaje. De esa
manera, el estudio en este punto desarrolla estrategias para promover
una educación inclusiva y justa en Ayacucho, respetando las diferencias
entre lo urbano y lo rural, y propone la flexibilidad de las propuestas
a las situaciones especiales de las zonas.

\subsection*{Preguntas de
investigación}\label{preguntas-de-investigaciuxf3n}
\addcontentsline{toc}{subsection}{Preguntas de investigación}

\begin{center}

\justify
\begin{enumerate}
    \item ¿Existen diferencias significativas en los niveles de educación entre las zonas urbanas y rurales de Ayacucho, según los resultados de la prueba t de Student?
    \item ¿Qué factores pueden explicar las diferencias en los niveles de educación entre las zonas rurales y urbanas, según los resultados de la prueba t de Student?
    \item ¿Cómo varían las habilidades de comprensión lectora y matemáticas de los estudiantes según su ubicación rural o urbana, y son estas diferencias estadísticamente significativas?
    \item Según los resultados de la prueba t de Student, ¿cómo predice el acceso a recursos tecnológicos el rendimiento académico de los estudiantes en las áreas rurales frente a las urbanas?
\end{enumerate}

\end{center}

\section*{Marco teórico}\label{marco-teuxf3rico}
\addcontentsline{toc}{section}{Marco teórico}

\subsection*{La Prueba t para Muestras Independientes en el Análisis de
la Desigualdad
Educativa}\label{la-prueba-t-para-muestras-independientes-en-el-anuxe1lisis-de-la-desigualdad-educativa}
\addcontentsline{toc}{subsection}{La Prueba t para Muestras
Independientes en el Análisis de la Desigualdad Educativa}

\vspace{0.5cm}

Según DATAtab Team (2024), la prueba t para muestras independientes es
una herramienta estadística adecuada para comparar las medias de dos
muestras independientes, en este caso, los estudiantes de zonas rurales
y urbanas. En este estudio, se utilizará la prueba t para evaluar si las
diferencias en las calificaciones de los estudiantes de segundo año de
secundaria de las zonas rurales y urbanas de Ayacucho son
estadísticamente significativas. Este método permite determinar si las
disparidades observadas en el rendimiento académico entre estos dos
grupos son suficientemente grandes como para descartar la posibilidad de
que sean producto del azar.

Los supuestos de la prueba t incluyen la independencia de las
observaciones, normalidad en la distribución de las calificaciones y
homogeneidad de varianzas. En caso de cumplir con estos supuestos, se
aplicará la fórmula de la prueba t para muestras independientes:

\[
t = \frac{X_1 - X_2}{\sqrt{\frac{S_1^2}{n_1} + \frac{S_2^2}{n_2}}}
\]

Donde:

\begin{itemize}
    \item $X_1$ y $X_2$ son las medias de los grupos urbano y rural.
    \item $S_1^2$ y $S_2^2$ representan las varianzas de las calificaciones.
    \item $n_1$ y $n_2$ son los tamaños muestrales de cada grupo.
\end{itemize}

El análisis estadístico basado en esta prueba permitirá validar o
rechazar la hipótesis nula, que establece que no existe diferencia
significativa en las calificaciones entre estudiantes urbanos y rurales.
En caso de que el valor de la t calculada sea mayor que el valor
crítico, se aceptará la hipótesis alternativa, indicando que sí hay una
diferencia significativa en el rendimiento académico de estos dos
grupos.

\subsection*{Desigualdad Educativa entre Zonas Rurales y Urbanas en
Perú}\label{desigualdad-educativa-entre-zonas-rurales-y-urbanas-en-peruxfa}
\addcontentsline{toc}{subsection}{Desigualdad Educativa entre Zonas
Rurales y Urbanas en Perú}

\vspace{0.5cm}

Uno de los principales problemas que enfrenta el sistema educativo
peruano es la marcada desigualdad entre las áreas rurales y urbanas.
Según Cueto (2004) (2004), los estudiantes rurales presentan
significativamente menores logros académicos en comparación con los
urbanos, debido a factores como la pobreza, la falta de acceso a
servicios básicos y la baja calidad de la educación en estas zonas. En
las áreas rurales, los estudiantes suelen tener el quechua como lengua
materna, lo que añade barreras adicionales en su proceso educativo.
Además, enfrentan el reto de aulas multigrado, en las cuales un solo
maestro instruye a alumnos de diversas edades y niveles. Estas
condiciones contrastan marcadamente con las de las áreas urbanas, donde
los estudiantes tienen acceso a mejores recursos y una estructura
educativa más favorable.

\subsection*{Políticas Educativas en Zonas Rurales: Educación Bilingüe
Intercultural}\label{poluxedticas-educativas-en-zonas-rurales-educaciuxf3n-bilinguxfce-intercultural}
\addcontentsline{toc}{subsection}{Políticas Educativas en Zonas Rurales:
Educación Bilingüe Intercultural}

\vspace{0.5cm}

El gobierno peruano ha desarrollado políticas para mejorar la educación
en zonas rurales, siendo el programa de Educación Bilingüe Intercultural
(EBI) uno de los más destacados. Este programa busca atender las
necesidades de estudiantes que hablan lenguas indígenas, como el
quechua. Sin embargo, Cueto (2004) (2004) señala que la implementación
de estas políticas ha sido insuficiente y poco adaptada a las realidades
locales, lo que limita su efectividad. Las intervenciones educativas
deben ir más allá del ámbito escolar, abordando también las necesidades
socioeconómicas de los estudiantes para lograr un impacto significativo
en la calidad educativa.

\subsection*{Factores Socioeconómicos y Rendimiento
Académico}\label{factores-socioeconuxf3micos-y-rendimiento-acaduxe9mico}
\addcontentsline{toc}{subsection}{Factores Socioeconómicos y Rendimiento
Académico}

\vspace{0.5cm}

Diversos estudios confirman que el rendimiento escolar en las zonas
rurales está influenciado por factores socioeconómicos, como el nivel de
pobreza y la disponibilidad de recursos educativos en el hogar. Cueto
(2004) (2004) destaca que aspectos familiares, como la educación de los
padres y el acceso a materiales educativos en el hogar, tienen un
impacto directo en el rendimiento académico. Adicionalmente, el
analfabetismo en muchos hogares rurales agrava la falta de apoyo para
los estudiantes, generando una desventaja que se refleja en los
resultados académicos. La malnutrición es otro factor importante,
afectando directamente el rendimiento de los estudiantes rurales,
quienes suelen vivir en condiciones de pobreza extrema y con altos
índices de desnutrición.

\subsection*{Infraestructura y Recursos Educativos en las Zonas
Rurales}\label{infraestructura-y-recursos-educativos-en-las-zonas-rurales}
\addcontentsline{toc}{subsection}{Infraestructura y Recursos Educativos
en las Zonas Rurales}

\vspace{0.5cm}

La falta de infraestructura adecuada es uno de los problemas más graves
que enfrentan las escuelas rurales en Perú. Cueto (2004) menciona que
muchas escuelas rurales carecen de servicios básicos, como electricidad
o agua potable, lo que limita enormemente el proceso de
enseñanza-aprendizaje. A diferencia de las áreas urbanas, donde la
infraestructura educativa es mejor, las escuelas rurales se ven
obligadas a operar en condiciones precarias, lo que contribuye al bajo
rendimiento de los estudiantes. La falta de programas de apoyo para
estudiantes con dificultades también perpetúa el rezago académico en
estas áreas.

\subsection*{Desigualdad Persistente en los Resultados de
Aprendizaje}\label{desigualdad-persistente-en-los-resultados-de-aprendizaje}
\addcontentsline{toc}{subsection}{Desigualdad Persistente en los
Resultados de Aprendizaje}

\vspace{0.5cm}

A pesar de los esfuerzos recientes para mejorar la educación en las
zonas rurales, la brecha en los resultados de aprendizaje entre áreas
rurales y urbanas persiste. Según el Consejo Nacional de Educación
(2016), las mejoras en el medio urbano están ocurriendo a un ritmo
considerablemente más acelerado que en el rural. La falta de acceso a
tecnologías educativas y el deficiente estado de la infraestructura
escolar rural siguen siendo barreras críticas. En zonas urbanas, seis de
cada diez escuelas primarias y siete de cada diez secundarias tienen
acceso a internet, mientras que en las zonas rurales, solo seis de cada
diez cuentan con electricidad, y cinco con agua potable (Consejo
Nacional de Educación, 2016).

\subsection*{Desafíos Lingüísticos en la Educación
Rural}\label{desafuxedos-linguxfcuxedsticos-en-la-educaciuxf3n-rural}
\addcontentsline{toc}{subsection}{Desafíos Lingüísticos en la Educación
Rural}

\vspace{0.5cm}

En muchas regiones rurales, como Ayacucho, una gran parte de la
población estudiantil tiene como lengua materna el quechua, lo que
genera desafíos adicionales en su proceso educativo. Aunque existen
programas como la Educación Bilingüe Intercultural, su implementación ha
sido limitada y no responde adecuadamente a las necesidades locales.
Cueto (2004) señala que el bajo dominio del castellano entre los
estudiantes rurales repercute directamente en su rendimiento en áreas
clave, como las matemáticas y la comprensión lectora, incrementando la
tasa de deserción escolar.

\subsection*{Condiciones Sociales y Educación
Rural}\label{condiciones-sociales-y-educaciuxf3n-rural}
\addcontentsline{toc}{subsection}{Condiciones Sociales y Educación
Rural}

\vspace{0.5cm}

Las condiciones sociales en las zonas rurales tienen un impacto directo
en la educación. La pobreza extrema, la desnutrición y las
responsabilidades laborales de los estudiantes limitan su capacidad de
asistir a la escuela y rendir adecuadamente. Además, las niñas enfrentan
barreras adicionales, como las expectativas culturales de asumir roles
domésticos o casarse a temprana edad, lo que contribuye a una mayor tasa
de deserción escolar entre las niñas en comparación con los niños.

\section*{Objetivos}\label{objetivos}
\addcontentsline{toc}{section}{Objetivos}

\subsection*{Objetivo general}\label{objetivo-general}
\addcontentsline{toc}{subsection}{Objetivo general}

\vspace{0.25cm}

Probar la hipótesis de si existe una diferencia en el nivel educativo
entre estudiantes de zonas rurales y urbanas en Ayacucho utilizando la
prueba t para muestras independientes.

\subsection*{Objetivos específicos}\label{objetivos-especuxedficos}
\addcontentsline{toc}{subsection}{Objetivos específicos}

\vspace{0.5cm}

\begin{itemize}
    \item Comparar si existe una diferencia significativa en el rendimiento académico de los estudiantes de zonas rurales y urbanas de Ayacucho, utilizando los resultados de la prueba t de Student.
    \item Explicar los factores responsables de la diferencia en el nivel de logro educativo entre los estudiantes rurales y urbanos en Ayacucho.
    \item Diseñar una prueba para los estudiantes rurales y urbanos, y evaluar la diferencia entre las medias de las calificaciones de ambas categorías.
    \item Comparar y contrastar el impacto del acceso a recursos tecnológicos en el rendimiento de los estudiantes en las zonas rurales frente a las urbanas utilizando los resultados de la prueba t.
    \item Observar cualquier diferencia significativa en el nivel educativo alcanzado por aquellos estudiantes con y sin acceso a programas bilingües en los entornos rurales y urbanos.
\end{itemize}

\section*{Hipótesis}\label{hipuxf3tesis}
\addcontentsline{toc}{section}{Hipótesis}

\subsection*{Hipótesis general}\label{hipuxf3tesis-general}
\addcontentsline{toc}{subsection}{Hipótesis general}

\begin{itemize}
    \item \textbf{H$_{0}$:} El rendimiento académico promedio en el área de matemáticas de los estudiantes de sexo femenino es igual al de los estudiantes de sexo masculino en zonas rurales de Ayacucho.

  \item \textbf{H$_{1}$:} SEl rendimiento académico promedio en el área de matemáticas de los estudiantes de sexo femenino es diferente al de los estudiantes de sexo masculino en zonas rurales de Ayacucho.

\end{itemize}

\subsection*{Hipótesis específicas}\label{hipuxf3tesis-especuxedficas}
\addcontentsline{toc}{subsection}{Hipótesis específicas}

\begin{itemize}
    \item \textbf{H$_{0}$:} No hay diferencia significativa en los puntajes de comprensión lectora entre estudiantes rurales y urbanos.
    \item \textbf{H$_{1}$:} Hay una diferencia significativa en los puntajes de comprensión lectora entre estudiantes rurales y urbanos.
    \item \textbf{H$_{0}$:} No hay diferencia significativa en los puntajes de matemáticas entre estudiantes rurales y urbanos.
    \item \textbf{H$_{1}$:} Hay una diferencia significativa en los puntajes de matemáticas entre estudiantes rurales y urbanos.
    \item \textbf{H$_{0}$:} El acceso a recursos tecnológicos no afecta de manera significativa el rendimiento académico de los estudiantes rurales y urbanos.
    \item \textbf{H$_{1}$:} El acceso a recursos tecnológicos sí afecta de manera significativa el rendimiento académico de los estudiantes rurales y urbanos.
    \item \textbf{H$_{0}$:} No hay una diferencia significativa en el nivel educativo entre estudiantes con y sin acceso a programas bilingües.
    \item \textbf{H$_{1}$:} Sí hay una diferencia significativa en el nivel educativo entre estudiantes con y sin acceso a programas bilingües.
\end{itemize}

\section*{Metodología}\label{metodologuxeda}
\addcontentsline{toc}{section}{Metodología}

\subsection*{Operacionalización de
variables}\label{operacionalizaciuxf3n-de-variables}
\addcontentsline{toc}{subsection}{Operacionalización de variables}

A continuación, se describen las principales variables del estudio,
indicando su significado y las formas en que serán medidas o
categorizadas.

\begin{itemize}
    \item sexo: Sexo del estudiante. Variable categórica con dos alternativas: masculino. Alternativas: Hombre, Mujer
    \item departamento: Departamento. Variable cualitativa que representa la región administrativa a la que pertenece la institución.
    \item M500-EM-2S-2023-MA: Medida en matemáticas. Variable cuantitativa que refleja el puntaje obtenido por los estudiantes en la evaluación de matemáticas.

\end{itemize}

\subsection*{Población}\label{poblaciuxf3n}
\addcontentsline{toc}{subsection}{Población}

La población de este estudio está constituida por todas las
instituciones educativas de la región de Ayacucho que imparten educación
básica regular.

Esto incluye escuelas públicas y privadas que presentan características
similares en términos de gestión y nivel educativo. La población se
delimita temporalmente al año académico 2023, lo que permite una
evaluación actualizada y pertinente del nivel educativo en el área de
matemáticas.

\subsubsection*{Normalidad de la
población}\label{normalidad-de-la-poblaciuxf3n}
\addcontentsline{toc}{subsubsection}{Normalidad de la población}

Primero debemos de comprobar que nuestra población sigue una
distribución normal.

\begin{Shaded}
\begin{Highlighting}[]
\FunctionTok{library}\NormalTok{(readxl)}
\FunctionTok{library}\NormalTok{(magrittr)}
\FunctionTok{library}\NormalTok{(dplyr)}
\NormalTok{BD }\OtherTok{\textless{}{-}} \FunctionTok{read\_excel}\NormalTok{(}\StringTok{"BD\_2S ENLA muestral 2023.xlsx"}\NormalTok{)}

\NormalTok{data\_rural }\OtherTok{\textless{}{-}}\NormalTok{ BD }\SpecialCharTok{\%\textgreater{}\%} 
  \FunctionTok{filter}\NormalTok{(departamento }\SpecialCharTok{==} \StringTok{"AYACUCHO"}\NormalTok{, area }\SpecialCharTok{==} \StringTok{"Rural"}\NormalTok{) }\SpecialCharTok{\%\textgreater{}\%} 
  \FunctionTok{select}\NormalTok{(}\FunctionTok{c}\NormalTok{(}\StringTok{"sexo"}\NormalTok{, }\StringTok{"M500\_EM\_2S\_2023\_MA"}\NormalTok{)) }\SpecialCharTok{\%\textgreater{}\%} 
  \FunctionTok{na.omit}\NormalTok{()}

\FunctionTok{shapiro.test}\NormalTok{(data\_rural}\SpecialCharTok{$}\NormalTok{M500\_EM\_2S\_2023\_MA)}
\end{Highlighting}
\end{Shaded}

\begin{verbatim}
## 
##  Shapiro-Wilk normality test
## 
## data:  data_rural$M500_EM_2S_2023_MA
## W = 0.9864, p-value = 2.823e-06
\end{verbatim}

Según el criterio de p\_valor, la población no sigue una distribución
normal.\\
Para detectar posibles datos atipicos vamos a usar el método z-score.

\begin{verbatim}
## 
##  Shapiro-Wilk normality test
## 
## data:  data_rural_sin$M500_EM_2S_2023_MA
## W = 0.998, p-value = 0.5628
\end{verbatim}

Nuevamente usando el test de \texttt{Shapiro-wilk} y por el criterio de
p\_valor comprobamos que el

\[p_{valor}=0.5628>\alpha=0.05\]\\
y se verifica que nuestra población ahora sigue una distribución normal.

\begin{center}\includegraphics{trabajo_final_analisis-estadistico_files/figure-latex/unnamed-chunk-3-1} \end{center}

\subsubsection*{Homogeneidad de
Varianzas}\label{homogeneidad-de-varianzas}
\addcontentsline{toc}{subsubsection}{Homogeneidad de Varianzas}

\[H_{0}: \text{Las varianzas de las notas entre hombres y mujeres son iguales.}\]\\
\[H_{0}: \text{Las varianzas de las notas entre hombres y mujeres son diferentes.}\]

\begin{verbatim}
## 
##  Bartlett test of homogeneity of variances
## 
## data:  data_rural_sin$M500_EM_2S_2023_MA by data_rural_sin$sexo
## Bartlett's K-squared = 0.21642, df = 1, p-value = 0.6418
\end{verbatim}

Dado que \(p_{valor}=0.6418>\alpha=0.05\), no se rechaza la hipótesis
nula. Esto implica que no hay evidencia estadísticamente significativa
para afirmar que las varianzas de las notas entre hombres y mujeres son
diferentes.

\begin{center}\includegraphics{trabajo_final_analisis-estadistico_files/figure-latex/unnamed-chunk-5-1} \end{center}

\begin{table}[!h]
\centering
\caption{\label{tab:Estadisticas genero}Resumen de las notas por género}
\centering
\begin{tabular}[t]{lrrrr}
\toprule
sexo & Media (\(\mu\)) & Varianza (\(\sigma^{2}\)) & Desviación Estándar (\(\sigma\)) & Tamaño de la población (\(N\))\\
\midrule
\cellcolor{gray!10}{Hombre} & \cellcolor{gray!10}{549.3282} & \cellcolor{gray!10}{4936.985} & \cellcolor{gray!10}{70.26368} & \cellcolor{gray!10}{342}\\
Mujer & 538.1761 & 4700.335 & 68.55899 & 380\\
\bottomrule
\end{tabular}
\end{table}

\subsection*{Muestra}\label{muestra}
\addcontentsline{toc}{subsection}{Muestra}

El uso de un diseño de muestreo estratificado permite dividir la
población en dos estratos claramente diferenciados: el masculino y el
femenino. Esta división se hace con el objetivo de que ambos génneros
estén justamente representados en la muestra, ya que se espera que sus
niveles de logros educativos sean similares.

Dentro de cada estrato, se realizará un muestreo aleatorio simple para
seleccionar a los estudiantes. De esta manera, se mantendrá la
aleatoriedad y se evitarán sesgos dentro de cada subgrupo.

\subsubsection*{Tamaño de la muestra}\label{tamauxf1o-de-la-muestra}
\addcontentsline{toc}{subsubsection}{Tamaño de la muestra}

Determinamos el tamaño de la muestra total por muestreo estratificado
(n) en 160 estudiantes.

Separamos nuestra población en dos estratos, ``masculino'' y
``femenino''.

Luego procedemos a estimar el tamaño de cada muestra de cada estrato.

\[n_{mujer}= n*\frac{N_{mujer}}{N}, \quad n_{hombre}= n*\frac{N_{hombre}}{N}, \quad \]

\begin{itemize}
    \item N: tamaño de la población 
    \item n: tamaño total de la muestra
    \item $N_{mujer}$: tamaño de la población solo de mujeres
    \item $N_{hombre}$: tamaño de la población solo de hombres
    \item $n_{mujer}$: tamaño de la muestra de mujeres
    \item $n_{hombre}$: tamaño de la muestra de hombre

\end{itemize}

\[n_{mujer}=84, \quad n_{hombre}=76 \]

\subsubsection*{Extracción de la
muestra}\label{extracciuxf3n-de-la-muestra}
\addcontentsline{toc}{subsubsection}{Extracción de la muestra}

En este paso usaremos el paquete \texttt{teachingsamplig} para extraer
nuestras muestras.

\begin{Shaded}
\begin{Highlighting}[]
\FunctionTok{library}\NormalTok{(TeachingSampling)}
\FunctionTok{set.seed}\NormalTok{(}\DecValTok{123}\NormalTok{)}
\NormalTok{sam }\OtherTok{\textless{}{-}} \FunctionTok{S.STSI}\NormalTok{(data\_rural\_sin}\SpecialCharTok{$}\NormalTok{sexo, }\FunctionTok{c}\NormalTok{(N\_h, N\_m), }\FunctionTok{c}\NormalTok{(n\_h, n\_m))}
\NormalTok{muestra }\OtherTok{\textless{}{-}}\NormalTok{ data\_rural\_sin[sam,]}

\NormalTok{muestra\_hombre }\OtherTok{\textless{}{-}}\NormalTok{ muestra }\SpecialCharTok{\%\textgreater{}\%} \FunctionTok{filter}\NormalTok{(sexo }\SpecialCharTok{==} \StringTok{"Hombre"}\NormalTok{) }\SpecialCharTok{\%\textgreater{}\%} \FunctionTok{select}\NormalTok{(}\DecValTok{2}\NormalTok{)}
\NormalTok{muestra\_mujer }\OtherTok{\textless{}{-}}\NormalTok{ muestra }\SpecialCharTok{\%\textgreater{}\%} \FunctionTok{filter}\NormalTok{(sexo }\SpecialCharTok{==} \StringTok{"Mujer"}\NormalTok{) }\SpecialCharTok{\%\textgreater{}\%} \FunctionTok{select}\NormalTok{(}\DecValTok{2}\NormalTok{)}
\end{Highlighting}
\end{Shaded}

\begin{table}[!h]
\centering
\caption{\label{tab:tabla}Resumen de cada muestra}
\centering
\begin{tabular}[t]{lrrrrr}
\toprule
sexo & Media (\(\bar{x}\)) & Varianza (\(s^{2}\)) & Desviación Estándar (\(s\)) & Tamaño (\(n\)) & Proporción (\(p\))\\
\midrule
\cellcolor{gray!10}{Hombre} & \cellcolor{gray!10}{553.216} & \cellcolor{gray!10}{5884.149} & \cellcolor{gray!10}{76.708} & \cellcolor{gray!10}{76} & \cellcolor{gray!10}{0.105}\\
Mujer & 538.551 & 5948.829 & 77.129 & 84 & 0.116\\
\bottomrule
\end{tabular}
\end{table}

\subsection*{Procedimiento}\label{procedimiento}
\addcontentsline{toc}{subsection}{Procedimiento}

\section*{Resultados}\label{resultados}
\addcontentsline{toc}{section}{Resultados}

\section*{Conclusiones}\label{conclusiones}
\addcontentsline{toc}{section}{Conclusiones}

\newpage

\section*{Referencias}\label{referencias}
\addcontentsline{toc}{section}{Referencias}

\phantomsection\label{refs}
\begin{CSLReferences}{1}{0}
\bibitem[\citeproctext]{ref-Cueto}
Cueto, S. (2004). \emph{Factores predictivos del rendimiento escolar,
deserci{ó}n e ingreso a educaci{ó}n secundaria en una muestra de
estudiantes de zonas rurales del per{ú}}.

\bibitem[\citeproctext]{ref-UNICEF}
Perú, UNICEF. R. en. (2016). \emph{Estado mundial de la infancia 2016:
Acabar con las inequidades para brindarle oportunidades justas a toda la
niñez}. UNICEF.

\end{CSLReferences}

\end{document}
